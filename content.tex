\begin{frame}[fragile]{Thompson's construction}
  \begin{description}
    \item[Algorithm] to construct an Non-deterministic Finite Automaton (NFA) from a regular expression.
    \item[NFA] will recognise the same language as the regular expression.
  \end{description}
  \vspace{6mm}
  \begin{alertblock}{Example: $a.b|b^*$}
    \vspace{2mm}
    \begin{adjustbox}{max width={0.8\textwidth}, center} 
      \begin{tikzpicture}[auto, on grid, node distance=24mm, initial text=, >=latex]

        \node[state, initial]            (q_1)                      {}; 
        \node[state]                     (q_2) [above right of=q_1] {};
        \node[state]                     (q_3) [below right of=q_1] {};

        \node[state]                     (q_4) [      right of=q_2] {};
        \node[state]                     (q_5) [      right of=q_4] {};
        \node[state]                     (q_6) [      right of=q_5] {};
        
        \node[state, node distance=36mm] (q_7) [      right of=q_3] {};
        \node[state, node distance=36mm] (q_8) [      right of=q_7] {};

        \node[state, accepting]          (q_9) [below right of=q_6] {}; 


        \path[->]
          (q_1) edge [below right]       node {$\epsilon$} (q_2)
                edge []                  node {$\epsilon$} (q_3)

          (q_2) edge []                  node {$\epsilon$} (q_4)
                edge [bend right, below] node {$\epsilon$} (q_6)
          (q_4) edge []                  node {$b$}        (q_5)
          (q_5) edge []                  node {$\epsilon$} (q_6)
                edge [bend left, below]  node {$\epsilon$} (q_4)
          (q_6) edge [below left]        node {$\epsilon$} (q_9)

          (q_3) edge []                  node {$a$}        (q_7)
          (q_7) edge []                  node {$b$}        (q_8)
          (q_8) edge []                  node {$\epsilon$} (q_9);

      \end{tikzpicture}
    \end{adjustbox}
  \end{alertblock}
\end{frame}


\begin{frame}[fragile]{Fragments}
  \begin{description}
    \item[Assume] the regular expression is in postfix.
    \item[Stack] of fragments of the overall NFA.
    \item[Normal] characters push to the stack.
    \item[Special] characters pop from and push to the stack.
  \end{description}
  \vspace{6mm}
  \begin{alertblock}{Example fragment}
    \vspace{2mm}
    \begin{adjustbox}{max width={0.8\textwidth}, center} 
      \begin{tikzpicture}[auto, on grid, node distance=24mm, initial text=, >=latex]

        \node[state, initial, fill=gmitblue]  (q_1) [] {};
        \node[state]                          (q_2) [right of=q_1] {};
        \node[state]                          (q_3) [right of=q_2] {};
        \node[state, accepting, fill=gmitred] (q_4) [right of=q_3] {};


        \path[->]
          (q_1) edge []                  node {$\epsilon$} (q_2)
                edge [bend right, below] node {$\epsilon$} (q_4)
          (q_2) edge []                  node {$b$}        (q_3)
          (q_3) edge []                  node {$\epsilon$} (q_4)
                edge [bend left, below]  node {$\epsilon$} (q_2);

        \begin{scope}[on background layer]
          \node[cloud, draw=black, fill=gmitgrey!20, cloud puffs=30, cloud puff arc=110, aspect=5, minimum width=68mm, minimum height=36mm] at (3.6,-0.3) {};
        \end{scope}
      \end{tikzpicture}
    \end{adjustbox}
  \end{alertblock}
\end{frame}


\begin{frame}[fragile]{Non-special characters}
  For a normal, non-special character $x$ push the following fragment to the stack.
  
  \vspace{4mm}
  
  \begin{adjustbox}{max width={0.8\textwidth}, center} 
    \begin{tikzpicture}[auto, on grid, node distance=24mm, initial text=, >=latex]
      \node[state, initial, fill=gmitblue]  (q_1) [] {};
      \node[state, accepting, fill=gmitred] (q_2) [right of=q_1] {};

      \path[->] (q_1) edge [] node {$x$} (q_2);
    \end{tikzpicture}
  \end{adjustbox}

  \vspace{4mm}

  We should include the empty regular expression $\epsilon$ too.
  
  \vspace{10mm}

  \begin{adjustbox}{max width={0.8\textwidth}, center} 
    \begin{tikzpicture}[auto, on grid, node distance=24mm, initial text=, >=latex]
      \node[state, initial, fill=gmitblue]  (q_1) [] {};
      \node[state, accepting, fill=gmitred] (q_2) [right of=q_1] {};

      \path[->] (q_1) edge [] node {$\epsilon$} (q_2);
    \end{tikzpicture}
  \end{adjustbox}
\end{frame}


\begin{frame}[fragile]{Concatenation $N.M$}
  When you see a $.$, pop two fragments from the stack and push the following instead.
  
  \vspace{18mm}
  
  \begin{adjustbox}{max width={\textwidth}, center}
    \begin{tikzpicture}[auto, on grid, node distance=24mm, initial text=, >=latex]
      
      \node[state, initial, fill=gmitblue]   (a_1)   []               {};
      \node[draw=none,fill=none]             (namea) [right of=a_1]   {Fragment 2};
      \node[state, fill=gmitred!50!gmitblue] (a_2)   [right of=namea] {};

      \node[draw=none,fill=none]             (nameb) [right of=a_2]   {Fragment 1};
      \node[state, fill=gmitred, accepting]  (b_2)   [right of=nameb] {};

      \begin{scope}[on background layer]
        \node[cloud, draw=black, fill=gmitgrey!20, cloud puffs=30, cloud puff arc=110, aspect=5, minimum width=50mm, minimum height=20mm, right of=a_1] {};
        \node[cloud, draw=black, fill=gmitgrey!20, cloud puffs=30, cloud puff arc=110, aspect=5, minimum width=50mm, minimum height=20mm, right of=a_2] {};
      \end{scope}
    \end{tikzpicture}
  \end{adjustbox}
\end{frame}


\begin{frame}[fragile]{Union $N|M$}
  When you see a $|$, pop two fragments from the stack and push the following instead.
  
  \vspace{4mm}
  
  \begin{adjustbox}{max width={0.8\textwidth}, center}
    \begin{tikzpicture}[auto, on grid, node distance=24mm, initial text=, >=latex]
      \node[state, initial, fill=gmitblue]  (s_i)   []                   {};
      
      \node[state, fill=gmitblue!50]        (a_1)   [above right of=s_i] {};
      \node[draw=none,fill=none]            (namea) [right of=a_1]       {Fragment 1};
      \node[state, fill=gmitred!50]         (a_2)   [right of=namea]     {};

      \node[state, fill=gmitblue!50]        (b_1)   [below right of=s_i] {};
      \node[draw=none,fill=none]            (nameb) [right of=b_1]       {Fragment 2};
      \node[state, fill=gmitred!50]         (b_2)   [right of=nameb]     {};

      \node[state, accepting, fill=gmitred] (s_a)   [below right of=a_2] {};

      \path[->] (s_i) edge [below right] node {$\epsilon$} (a_1)
                      edge [above right] node {$\epsilon$} (b_1)
                (a_2) edge [below left]  node {$\epsilon$} (s_a)
                (b_2) edge []            node {$\epsilon$} (s_a);
              

      \begin{scope}[on background layer]
        \node[cloud, draw=black, fill=gmitgrey!20, cloud puffs=30, cloud puff arc=110, aspect=5, minimum width=50mm, minimum height=20mm, right of=b_1] {};
        \node[cloud, draw=black, fill=gmitgrey!20, cloud puffs=30, cloud puff arc=110, aspect=5, minimum width=50mm, minimum height=20mm, right of=a_1] {};
      \end{scope}
    \end{tikzpicture}
  \end{adjustbox}
\end{frame}

\begin{frame}[fragile]{Kleene star $N^*$}
  When you see a $^*$, pop a fragment from the stack and push the following instead.
  
  \vspace{10mm}
  
  \begin{adjustbox}{max width={0.8\textwidth}, center}
    \begin{tikzpicture}[auto, on grid, node distance=24mm, initial text=, >=latex]
      \node[state, initial, fill=gmitblue]  (s_i)   []               {};
      
      \node[state, fill=gmitblue!50]        (a_1)   [right of=s_i]   {};
      \node[draw=none,fill=none]            (namea) [right of=a_1]   {Fragment};
      \node[state, fill=gmitred!50]         (a_2)   [right of=namea] {};

      \node[state, accepting, fill=gmitred] (s_a)   [right of=a_2]   {};

      \path[->] (s_i) edge []                     node {$\epsilon$} (a_1)
                      edge [bend right=40, below] node {$\epsilon$} (s_a)
                (a_2) edge []                     node {$\epsilon$} (s_a)
                      edge [bend right=90, above] node {$\epsilon$} (a_1);
              

      \begin{scope}[on background layer]
        \node[cloud, draw=black, fill=gmitgrey!20, cloud puffs=30, cloud puff arc=110, aspect=5, minimum width=50mm, minimum height=20mm, right of=a_1] {};
      \end{scope}
    \end{tikzpicture}
  \end{adjustbox}
\end{frame}


\begin{frame}{Data structures}
  Recall the definition of an NFA.
  \begin{description}
    \item[$Q$] is a finite set of \emph{states},
    \item[$\Sigma$] is a finite set called the \emph{alphabet},
    \item[$\delta$] is the \emph{transition function} ($Q \times \Sigma_{\epsilon} \rightarrow \mathcal{P}(Q)$),
    \item[$q_0$] is the \emph{start state} ($\in Q$), and
    \item[$F$] is the set of \emph{accept states} ($\subseteq Q$). 
  \end{description}

  \begin{alertblock}{Notes}
    \begin{itemize}
      \item Only need to know $\delta$, $q_0$ and $F$, and $|F| = 1$.
      \item Nothing points at $q_0$ and $q_f$ points at nothing.
      \item From every state is a single symbol arrow, or one or two $\epsilon$ arrows.
    \end{itemize}
  \end{alertblock}
\end{frame} 